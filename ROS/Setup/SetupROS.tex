\documentclass{scrartcl}

\usepackage{vhistory}  % version history at end of document
\usepackage{hyperref}  % hyper references

\usepackage{graphicx}
\graphicspath{ {./Images/} }

\newcommand{\docTitle}{Setup ROS}
\newcommand{\MEK}{Matt Kontz}

\hypersetup{%
 	pdftitle = {\docTitle},
	pdfkeywords = {\docTitle, Version \vhCurrentVersion
	from \vhCurrentDate},
	pdfauthor = {\vhListAllAuthorsLong}
}

\usepackage{scrpage2}
\pagestyle{scrheadings}
\ihead{\docTitle\ -- Version \vhCurrentVersion}
\chead[]{}
%\ohead[\thepage]{\thepage}
%\ifoot{\docTitle\ -- Version \vhCurrentVersion}
\cfoot[\thepage]{\thepage}
%\ofoot[\thepage]{\thepage}

\begin{document}

\title{\docTitle}
\author{\vhListAllAuthors}
\date{Version \vhCurrentVersion\ from \vhCurrentDate}
\maketitle

\section{Summary}
This document summizes how to setup ROS.  It will be assume that it is installed in an Ubuntu environment.

\section{About ROS}
\label{sec:install_ros}

From \href{http://www.ros.org/about-ros/}{www.ros.org} ...

\textit{The Robot Operating System (ROS) is a flexible framework for writing robot software. It is a collection of tools, libraries, and conventions that aim to simplify the task of creating complex and robust robot behavior across a wide variety of robotic platforms.}

\section{Install ROS}
\label{sec:install_ros}

Setup computer or VM with an OS (e.g. 16.04 - Xenial)) that matches the desired version of ROS (e.g. Kinitic Kane)

Find installation instrunction on \href{http://www.ros.org/about-ros/}{www.ros.org} for desired version of ROS and operating system (e.g. Kinitic and Ubuntu 16.04 (Xenial)) 

\begin{itemize}
	\item Choose Full or basic installation
	\item Initialize rosdep
	\item Follow instructions for environment setup
\end{itemize} 

\section{ROS Tutorials}
\label{sec:install_ros}

\begin{itemize}
	\item \href{http://wiki.ros.org/ROS/Tutorials}{wiki.ros.org/ROS/Tutorials}
	\item \href{https://www.udemy.com}{/www.udemy.com}
\end{itemize}

\begin{versionhistory}
\vhEntry{1.0}{Dec 8, 2018}{\MEK}{Created initial version with brief directions on setting up a Ubuntu VirtualBox virtual machine in a windows host.}

\end{versionhistory}

\end{document}
\grid
